\documentclass{article}

\usepackage{amsmath}
\usepackage{amsfonts}
\usepackage{url}
\usepackage{bbm} %for bold indicator func

\title{A More Detailed Proof of Prop. 9.5.12}

\begin{document}
\maketitle

This applies 9.5.9 to random variables that no longer have a (conditional) mean of $0$. Set $U_{N,j} = V_{N,j} - E[V_{N,j} \mid \mathcal{F}^N]$ and use 9.5.9. The first two conditions are easily checked. To check the third, we must show that 
\[
\sum_{j=1}^{M_N} E\left[U_{N,j}^2 \mathbbm{1} \left( |U_{N,j}| \ge \epsilon \right) \mid \mathcal{F}^N \right] \overset{P}{\to} 0.
\]

When they say ``by Jensen's," there's a little more to it.
\begin{align*}
U_{N,j}^2 &= \left(  V_{N,j} - E[V_{N,j} \mid \mathcal{F}^N] \right)^2 \\
&= V_{N,j}^2 - 2V_{N,j}E[V_{N,j} \mid \mathcal{F}^N] + \left( E[V_{N,j} \mid \mathcal{F}^N] \right)^2 \\
&= 2V_{N,j}^2 + 2 \left( E[V_{N,j} \mid \mathcal{F}^N] \right)^2 - \left(V_{N,j} +E[V_{N,j} \mid \mathcal{F}^N]  \right)^2 \\
&\le 2V_{N,j}^2 + 2 \left( E[V_{N,j} \mid \mathcal{F}^N] \right)^2  \\
&\le 2V_{N,j}^2 + 2  E[V_{N,j}^2 \mid \mathcal{F}^N] \tag{Jensen's}.
\end{align*}

This implies that 
\begin{align*}
\{|U_{N,j}| \ge \epsilon \} &= \{U_{N,j}^2 \ge \epsilon^2 \} \\
&\subset \{V_{N,j}^2 +   E[V_{N,j}^2 \mid \mathcal{F}^N] \ge \epsilon^2/2 \} \\
&\subset \{V_{N,j}^2 \ge \epsilon^2/4\}  \cup  \{ E[V_{N,j}^2 \mid \mathcal{F}^N] \ge \epsilon^2/4 \}
\end{align*}
which in turn implies that 
\[
\mathbbm{1} \left( |U_{N,j}| \ge \epsilon \right) \le \mathbbm{1} \left( V_{N,j}^2 \ge \epsilon^2/4 \right) + \mathbbm{1}\left( E[V_{N,j}^2 \mid \mathcal{F}^N] \ge \epsilon^2/4 \right)
\]
and taking these two inequalities together we have
\[
U_{N,j}^2 \mathbbm{1} \left( |U_{N,j}| \ge \epsilon \right) \le 2\left( V_{N,j}^2 +   E[V_{N,j}^2 \mid \mathcal{F}^N]\right) \left( \mathbbm{1} \left( \{V_{N,j}^2 \ge \epsilon^2/4 \right) + \mathbbm{1}\left(  E[V_{N,j}^2 \mid \mathcal{F}^N] \ge \epsilon^2/4 \right) \right).
\]

Take the conditional expectation on both sides, then sum. 
\begin{align*}
&\sum_j E[U_{N,j}^2 \mathbbm{1} \left( |U_{N,j}| \ge \epsilon \right)  ] \\
&\le 2 \sum_j E[ V_{N,j}^2  \mathbbm{1}\left( V_{N,j}^2 \ge \epsilon^2/4 \right) \mid \mathcal{F}^N] + 2 \sum_j E[ V_{N,j}^2 \mid \mathcal{F}^N] \mathbbm{1}\left( E[V_{N,j}^2 \mid \mathcal{F}^N] \ge \epsilon^2/4 \right) \\
& \hspace{10mm} +  2 \sum_j E[V_{N,j}^2 \mid \mathcal{F}^N]  P( V_{N,j}^2 \ge \epsilon^2/4 \mid \mathcal{F}^N )  + 2 \sum_j E[V_{N,j}^2 \mid \mathcal{F}^N]  \mathbbm{1}\left( E[V_{N,j}^2 \mid \mathcal{F}^N] \ge \epsilon^2/4 \right) \\
&= 2 \sum_j E[ V_{N,j}^2 \mathbbm{1} \left( V_{N,j}^2 \ge \epsilon^2/4 \right) \mid \mathcal{F}^N] + 4 \sum_j E[ V_{N,j}^2 \mid \mathcal{F}^N] \mathbbm{1}\left(  E[V_{N,j}^2 \mid \mathcal{F}^N] \ge \epsilon^2/4 \right) \\
& \hspace{10mm} +  2 \sum_j E[V_{N,j}^2 \mid \mathcal{F}^N]  P( V_{N,j}^2 \ge \epsilon^2/4 \mid \mathcal{F}^N ) \\
&=2 \sum_j E[ V_{N,j}^2 \mathbbm{1}\left( V_{N,j}^2 \ge \epsilon^2/4 \right) \mid \mathcal{F}^N] +  4 B_N + 2A_N \tag{defns} \\
&\to 0 \tag{9.5.7 and assumption}.
\end{align*}

The first term goes to $0$ assumption (iii) of this proposition. The other two terms go to $0$ using logic that is similar to the tools we used at the end of proof 9.5.7. We will reproduce those details here, but recall that in 9.5.7, there was an inequality trick we discussed. For details on that inequality, see \verb|prof_9.5.7_proof.pdf|.

To prove $A_N \overset{p}{\to} 0$: 

\begin{align*}
A_n &= \sum_i E[ V_{N,i}^2  \mid \mathcal{F}^N] P(V_{N,i}^2 \ge \epsilon^2/4 \mid \mathcal{F}^N ) \\
&\le P(\max_j V_{N,j}^2 \ge \epsilon^2/4 \mid \mathcal{F}^N ) \sum_i E[ V_{N,i}^2  \mid \mathcal{F}^N] \tag{max} \\
&\le P\left( \left\{ \epsilon^2/8 + \sum_{j=1}^{M_N} V_{N,j}^2 \mathbbm{1}(V_{N,j}^2 \ge \epsilon^2/8) \right\}  \ge \epsilon^2/4 \bigg\rvert \mathcal{F}^N \right) \sum_i E[ V_{N,i}^2  \mid \mathcal{F}^N]  \tag{9.5.7's second trick} \\
&= P\left( \sum_{j=1}^{M_N} V_{N,j}^2 \mathbbm{1}(V_{N,j}^2 \ge \epsilon^2/8)   \ge \epsilon^2/8 \bigg\rvert \mathcal{F}^N \right) \sum_i E[  V_{N,i}^2  \mid \mathcal{F}^N]   \\
&\le (8/ \epsilon^2) \sum_j E\left[ V_{N,j}^2 \mathbbm{1}(V_{N,j}^2 \ge \epsilon^2/8) \mid \mathcal{F}^N \right]  \sum_i E[ V_{N,i}^2  \mid \mathcal{F}^N]  \tag{Markov's}.
\end{align*}
$\sum_i E\left[ V_{N,i}^2 \mathbbm{1}(V_{N,i}^2 \ge \epsilon^2/8) \mid \mathcal{F}^N \right]$ goes to $0$ by hypothesis. If we can show that the other term is bounded in probability, then the whole thing goes to $0$ by Lemma 9.5.3 (3). TODO

% Proving $B_N \overset{p}{\to} 0$: 
% \newline
% 
% This uses the second inequality trick again:
% \begin{align*}
% B_N &=   \sum_i E[ | V_{N,i}|  \mid \mathcal{F}^N]  \mathbbm{1}( E[|V_{N,i}| \mid \mathcal{F}^N ] \ge \epsilon/2) \\
% &\le  \mathbbm{1}( \max_i E[|V_{N,i}| \mid \mathcal{F}^N ] \ge \epsilon/2) \sum_i E[ | V_{N,i}|  \mid \mathcal{F}^N] \tag{max} \\
% &\le  \mathbbm{1} \left\{ \sum_i E[|V_{N,i}| \mathbbm{1}(| V_{N,i} |\ge \epsilon/4) \mid \mathcal{F}^N ] \ge \epsilon/4 \right\}\sum_i E[ | V_{N,i}|  \mid \mathcal{F}^N]  \tag{second trick} \\
% &\le (4/\epsilon) \sum_i E[|V_{N,i}| \mathbbm{1}(| V_{N,i} |\ge \epsilon/4) \mid \mathcal{F}^N ]  \sum_i E[ | V_{N,i}|  \mid \mathcal{F}^N] \tag{logic} \\
% &\overset{p}{\to} 0
% \end{align*}
% where the last line follows because $\sum_i E[|V_{N,i}| \mathbbm{1}(| V_{N,i} |\ge \epsilon/4) \mid \mathcal{F}^N ]$ goes to $0$ by hypothesis, and the second is bounded in probability by hypothesis, which allows us to use Lemma 9.5.3(3) again. 
% 
% In the second to last line, I say ``logic" because this sum is nonnegative, and either it is bigger than $\epsilon/4$, or it isn't--verify the inequality holds in these two cases. 


% 
% 
% 
% To show $A_N$ goes to $0$, we show $P( V_{N,j}^2 \ge \epsilon^2/4 \mid \mathcal{F}^N )$ goes to zero:
% \begin{align*}
% P( V_{N,j}^2 \ge \epsilon^2/4 \mid \mathcal{F}^N ) &\le P( \max_j V_{N,j}^2 \ge \epsilon^2/4 \mid \mathcal{F}^N ) \\
% &\le P( \sum_j V_{N,j}^2 1_{\{V_{N,j}^2 \ge \epsilon^2 / 8 \} } \ge \epsilon^2/8 \mid \mathcal{F}^N ) \\
% &\le 8/\epsilon^2 E\left[ \sum_j V_{N,j}^2 1_{\{V_{N,j}^2 \ge \epsilon^2 / 8 \} } \mid \mathcal{F}^N \right]   \tag{Markov's} \\
% &\to 0 \tag{assumption iii}.
% \end{align*}

To show $B_N$ goes to zero:
\begin{align*}
B_N &= \sum_j E[ V_{N,j}^2 \mid \mathcal{F}^N] \mathbbm{1}\left(  E[V_{N,j}^2 \mid \mathcal{F}^N] \ge \epsilon^2/4 \right) \\
&\le \mathbbm{1} \left( \max_j E[V_{N,j}^2 \mid \mathcal{F}^N] \ge \epsilon^2/4 \right) \sum_j E[ V_{N,j}^2 \mid \mathcal{F}^N] \\
&\le \mathbbm{1} \left( \sum_j E[ V_{N,j}^2 1(|V_{N,j}| > \epsilon^2/8) \mid \mathcal{F}^N] \ge \epsilon^2/8 \right) \sum_j E[ V_{N,j}^2 \mid \mathcal{F}^N] \\
&\le (8/\epsilon^2)\sum_j E[ V_{N,j}^2 \mathbbm{1} \left(|V_{N,j}| > \epsilon^2/8 \right) \mid \mathcal{F}^N] \sum_j E[ V_{N,j}^2 \mid \mathcal{F}^N] \\
&\to 0 \tag{assumption iii and ii}.
\end{align*}

The last line follows because $\sum_j E[ V_{N,j}^2 \mathbbm{1} \left(|V_{N,j}| > \epsilon^2/8 \right) \mid \mathcal{F}^N]$ converges to $0$ by hypothesis, and because we showed in the previous step that $\sum_j E[ V_{N,j}^2 \mid \mathcal{F}^N]$ is bounded in probability.

This ensures condition (iii) of 9.5.9, and therefore the conclusion holds.




\end{document}