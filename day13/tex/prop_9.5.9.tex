\documentclass{article}

\usepackage{amsmath}
\usepackage{amsfonts}
\usepackage{url}
\usepackage{bbm} %for bold indicator func

\title{A More Detailed Proof of Prop. 9.5.9}

\begin{document}
\maketitle

Take care to realize that the book has two $i$'s. One represents the complex number $i = \sqrt{-1}$, and the other represents the index of the sum. To prevent confusion, the book changed the font for the complex number. However, I choose a different path, and I use $j$ as the index of the sum. 
\newline

The goal is to show that 
\[
E\left[ \exp\left( i u \sum_j U_{N,j} \right) \bigg\rvert \mathcal{F}^N \right] - \exp(- \sigma^2 u^2/2) \overset{p}{\to} 0.
\]

It is equivalent to show
\[
\underbrace{
E\left[ \exp\left( i u \sum_j U_{N,j} \right) \bigg\rvert \mathcal{F}^N \right] 
- \exp\left( -(u^2/2) \sum_j \sigma^2_{N,j}  \right)}_{\text{first chunk}}
\underbrace{
+ \exp\left( -(u^2/2) \sum_j \sigma^2_{N,j}  \right)
- \exp(- \sigma^2 u^2/2)}_{\text{second chunk}} 
\]
converges in probability to $0$. 

However, the second chunk is easy...by the continuous mapping theorem, together with assumption (ii), we have
\[
\exp\left( -(u^2/2) \sum_j \sigma^2_{N,j}  \right) - \exp(- \sigma^2 u^2/2)\overset{p}{\to} 0.
\]

So this explains why it suffices to prove 9.74 (the left chunk).

We provided Lemma 9.5.8 for complex numbers with modulus less than $1$. Keep in mind that real numbers are complex numbers, too, so we can apply this theorem there as well.


\begin{align*}
&  \left| E\left[\exp\left( i u \sum_j U_{N,j} \right) -  \exp\left( -(u^2/2) \sum_j \sigma^2_{N,j}  \right)  \bigg\rvert \mathcal{F}^N \right] \right| \\
&=  \left| E\left[ \prod_j \exp\left( i u  U_{N,j} \right) -  \prod_j \exp\left( -(u^2/2) \sigma^2_{N,j}  \right)  \bigg\rvert \mathcal{F}^N \right] \right| \tag{algebra}\\
&=  \left|  \prod_j E\left[ \exp\left( i u  U_{N,j} \right)\bigg\rvert \mathcal{F}^N \right] -  \prod_j E\left[\exp\left( -(u^2/2) \sigma^2_{N,j}  \right)  \bigg\rvert \mathcal{F}^N \right] \right| \tag{independence, linearity} \\
&\le  \sum_j \left| E\left[  \exp\left( i u  U_{N,j} \right) - \exp\left( -(u^2/2) \sigma^2_{N,j}  \right)  \bigg\rvert \mathcal{F}^N \right] \right| \tag{Lemma 9.5.8, Linearity}\\
&=  \sum_j \left| E\left[  \exp\left( i u  U_{N,j} \right) - (1 - u^2 \sigma^2_{N,j}/2) + (1 - u^2 \sigma^2_{N,j}/2) -\exp\left( -(u^2/2) \sigma^2_{N,j}  \right)  \bigg\rvert \mathcal{F}^N \right] \right|  \tag{algebra}\\
&\le \sum_j \left| E\left[   \exp\left( i u U_{N,j} \right)  - (1 - u^2 \sigma^2_{N,j}/2) \mid \mathcal{F}^N \right] \right|  \\
&\hspace{10mm} + \sum_j  \left|E\left[  \exp\left( -(u^2/2) \sigma^2_{N,j}  \right) - (1 - u^2 \sigma^2_{N,j}/2)   \mid \mathcal{F}^N \right] \right| \tag{linearity, tri-ineq} \\
&\le \sum_j \left| E\left[   \exp\left( i u U_{N,j} \right)  - (1 - u^2 \sigma^2_{N,j}/2) \mid \mathcal{F}^N \right] \right|  \\
&\hspace{10mm} + \sum_j  \left|  \exp\left( -(u^2/2) \sigma^2_{N,j}  \right) - (1 - u^2 \sigma^2_{N,j}/2)    \right| \\
&= A_N + B_N. \tag{defn}
\end{align*}

So, the overall goal is to show these two terms converge to $0$. 
Focusing on $A_N$ first, we will use the inequality we proved in the last section, namely 

\[
-\min\{| u U_{N,j}|^2, | u U_{N,j}|^3 \} \le e^{i u U_{N,j}} - (1 +i u U_{N,j} - \frac{1}{2} u^2 U_{N,j}^2) \le \min\{| u U_{N,j}|^2, | u U_{N,j}|^3 \}.
\]

Take conditional expectations on all sides, then take the absolute value (in that order). 

This gives you the book's inequality (the third one, unnumbered, on page 340):

\begin{align*}
\left| E\left[e^{i u U_{N,j}} - (1 +i u U_{N,j} - \frac{1}{2} u^2 U_{N,j}^2) \mid \mathcal{F}^N \right]  \right| 
&= \left| E\left[e^{i u U_{N,j}}\mid \mathcal{F}^N \right] - (1  - \frac{1}{2} u^2 \sigma^2_{N,j})   \right|\\
&\le E\left[ \min\{| u U_{N,j}|^2, | u U_{N,j}|^3 \} \mid \mathcal{F}^N \right] \\
&\le E\left[ | u U_{N,j}|^2 \mid \mathcal{F}^N \right] \\
&<\infty \tag{assumption}.
\end{align*}

We are using the fact that the mean of each $U_{N,j}$ is zero. The absolute value sign has to be on the outside.

Another trick we need is the following inequality. Let $\epsilon > 0$. Then
\begin{align*}
&\min\{|uU_{N,j} |^2, |uU_{N,j}|^3 \} \\
&=\min\{|uU_{N,j} |^2, |uU_{N,j}|^3 \}  1(|U_{N,j}| > \epsilon) + \min\{|uU_{N,j} |^2, |uU_{N,j}|^3 \}  1(|U_{N,j}| \le \epsilon) \\
&\le |uU_{N,j} |^2 1(|U_{N,j}| > \epsilon) + \epsilon|u|^3 |U_{N,j}|^2  \mathbbm{1}(|U_{N,j}| \le \epsilon) \\
&\le |uU_{N,j} |^2 1(|U_{N,j}| > \epsilon) + \epsilon|u|^3 |U_{N,j}|^2. 
\end{align*}


This let's us take care of the first part:
\begin{align*}
A_N &= \sum_j \left| E\left[   \exp\left( i u U_{N,j} \right)  - (1 - u^2 \sigma^2_{N,j}/2) \mid \mathcal{F}^N \right] \right| \tag{defn.} \\
&\le \sum_{j=1}^{M_N} E[ \min\{|uU_{N,j} |^2, |uU_{N,i}|^3 \} \mid \mathcal{F}^N ] \tag{Taylor's ineq} \\
&\le u^2 \sum_{j=1}^{M_N} E[ U_{N,j}^2 \mathbbm{1}(|U_{N,j}| > \epsilon) \mid \mathcal{F}^N ] + \epsilon |u|^3 \sum_{j=1}^{M_N} \sigma^2_{N,j} \tag{above inequality} \\
&\to 0 + \epsilon |u|^3 \sigma^2. \tag{assumption ii. and iii.}
\end{align*}

Now we look at $B_n$. The real-valued version of Taylor's Inequality states
\[
\bigg\rvert e^{-x} - \sum_{i=0}^n \frac{(-x)^i}{i!}\bigg\rvert \le \frac{M}{(n+1)!} | x|^{n+1}
\]
as long as $|f^{(n+1)}(x)| \le M$. Set $n=1$, and looking at $x \ge 0$, which means $|f^{(n+1)}(x)| = |f^{(2)}(x)| = |e^{-x}| \le 1$. This gives us, 
\[
|e^{-x} - 1 + x  | \le x^2/2.
\]
What does this mean for $B_N$? Well,
\begin{align*}
B_N &= \sum_{i=1}^{M_N} |\exp(-u^2 \sigma^2_{N,i}/2) - (1-u^2 \sigma^2_{N,i}/2)| \tag{defn.} \\
&\le \sum_{i=1}^{M_N} \left(u^2 \sigma^2_{N,i}/2 \right)^2/2 \tag{real-valued Taylor's}\\
&= \frac{u^4}{8} \sum_{i=1}^{M_N} \sigma^4_{N,i} \\
&\le \frac{u^4}{8} \left(\max_i \sigma^2_{N,i} \right)\sum_{i=1}^{M_N} \sigma^2_{N,i} \tag{see below} \\
&\to \frac{u^4}{8} \times 0 \times \sigma^2.
\end{align*}
The last line follows because of assumption 2 and the Uniform Smallness Condition (Remark 9.5.10.) 

\end{document}