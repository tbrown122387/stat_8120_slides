\documentclass{article}

\usepackage{amsmath}
\usepackage{amsfonts}
\usepackage{url}
\usepackage{bbm} %for bold indicator func

\title{A More Detailed Proof of Taylor's Inequality (page 340 in IHMM)}

\begin{document}
\maketitle


IHMM's ``Taylor's Inequality" is a special case of the following, which is from page 343 of Billingsley: 
\[
\left|e^{ix} - \sum_{k=0}^n\frac{(ix)^k}{k!}\right| \le \min\left\{\frac{|x|^{n+1} }{(n+1)!}, \frac{2|x|^n}{n!} \right\}.
\]

The main tool we use to prove this inequality is the following result that comes from integration by parts. To reiterate what Billingsley says,
\begin{align*}
\int_0^x (x-s)^n e^{is}ds &= -\frac{(x-s)^{n+1} }{n+1} e^{is} \bigg\rvert_{s=0}^{s=x} + \int_0^x  \frac{i(x-s)^{n+1} }{n+1} e^{is} ds  \\
&= \left[0+ \frac{x^{n+1} }{n+1}\right] + \int_0^x  \frac{i(x-s)^{n+1} }{n+1} e^{is} ds  \\
&=  \frac{x^{n+1} }{n+1} + \frac{i}{n+1}  \int_0^x  (x-s)^{n+1}e^{is} ds.
\end{align*}
To prove this statement is true for any $n \ge 0$, use induction. Start at $n=0$ and then go up. To illustrate that induction, consider the following:
\begin{align*}
i^{-1}(e^{ix}-1) &= \int_0^x  e^{is}ds \\
&= x + i  \int_0^x  (x-s) e^{is} ds \tag{IBP}\\
&= x + i  \left(\frac{x^2}{2} + \frac{i}{2} \int_0^x  (x-s)^2 e^{is} \right) ds \tag{IBP} \\
&= x + i\frac{x^2}{2} - \frac{1}{2} \int_0^x  (x-s)^2 e^{is}  ds  \\
&= x + i\frac{x^2}{2} - \frac{1}{2} \left(  \frac{x^3}{3} +\frac{i}{3} \int_0^x (x-s)^3 e^{is}  ds \right) \tag{IBP}\\
&= x + i\frac{x^2}{2} - \frac{x^3}{3!} - \frac{i}{3!} \int_0^x (x-s)^3 e^{is}  ds \\
&= i^0 x + i^1 \frac{x^2}{2}  + i^2 \frac{x^3}{3!} +  \frac{i^3}{3!} \int_0^x (x-s)^3 e^{is}  ds \\
&\vdots \\
&= \sum_{k=1}^{n} \frac{ i^{k-1}x^k }{k!} + \frac{i^n}{n!}\int_0^x (x-s)^n e^{is}  ds.
\end{align*}

Equation (26.2) in Billingsley comes from the above after solving for $e^{ix}$:

\begin{align*}
e^{ix} &= 1 + \sum_{k=1}^{n} \frac{ i^{k}x^k }{k!} + \frac{i^{n+1} }{n!}\int_0^x (x-s)^n e^{is}  ds \\
&= \sum_{k=0}^{n} \frac{ i^{k}x^k }{k!} + \frac{i^{n+1} }{n!}\int_0^x (x-s)^n e^{is}  ds \tag{1}.
\end{align*}

We also need a variant of this equation:

\begin{align*}
e^{ix} &= \sum_{k=0}^{n-1} \frac{ i^{k}x^k }{k!} + \frac{i^{n} }{(n-1)!}\int_0^x (x-s)^{n-1} (e^{is} -1 + 1)  ds \tag{replace $n$ with $n-1$} \\
&= \sum_{k=0}^{n-1} \frac{ i^{k}x^k }{k!} + \frac{i^{n} }{(n-1)!}\int_0^x (x-s)^{n-1} (e^{is} -1)  ds + \frac{i^{n} }{(n-1)!}\int_0^x (x-s)^{n-1}ds \\
&= \sum_{k=0}^{n-1} \frac{ i^{k}x^k }{k!} + \frac{i^{n} }{(n-1)!}\int_0^x (x-s)^{n-1} (e^{is} -1)  ds + \frac{i^{n} }{(n-1)!}\frac{x^n}{n} \tag{substitution} \\
&= \sum_{k=0}^{n} \frac{ i^{k}x^k }{k!} + \frac{i^{n} }{(n-1)!}\int_0^x (x-s)^{n-1} (e^{is} -1)  ds. \tag{2}
\end{align*}

We use (1) when $x \ge 0$, and we use (2) when $x < 0$. When $x \ge 0$:
\begin{align*}
\left| e^{ix} - \sum_{k=0}^{n} \frac{ i^{k}x^k }{k!} \right| &= \left| \frac{i^{n+1} }{n!}\int_0^x (x-s)^n e^{is}  ds \right| \\
&\le \frac{1 }{n!}\int_0^x |(x-s)^n e^{is}|  ds \tag{Jensen's} \\
&\le \frac{1 }{n!}\int_0^x (x-s)^n   ds \\
&= \frac{x^{n+1}}{(n+1)!} \tag{substitution}.
\end{align*}

When $x < 0$, 
\begin{align*}
\left| e^{ix} - \sum_{k=0}^{n} \frac{ i^{k}x^k }{k!} \right| &\le \frac{1 }{(n-1)!}\int_0^x |(x-s)^{n-1}| |e^{is} -1|  ds \tag{Jensen's}  \\
&= \frac{1 }{(n-1)!}\int_0^x ( s - x)^{n-1} |e^{is} -1|  ds \tag{ $x < 0$ } \\
&\le \frac{2 }{(n-1)!}\int_0^x (s-x)^{n-1}   ds \tag{tri-ineq} \\
&= -\frac{2 }{(n-1)!}\int_x^0 (s-x)^{n-1}   ds \\
&= \frac{2 |x|^n}{n!}\tag{substitution}
\end{align*}

Since they are both true
\[
\left| e^{ix} - \sum_{k=0}^{n} \frac{ i^{k}x^k }{k!} \right| \le \min \left\{\frac{|x|^{n+1}}{(n+1)!}  , \frac{2 |x|^n}{n!} \right\}
\]
and in the case of when $n=2$
\[
\left| e^{ix} - (1 + ix -x^2/2)  \right| \le \min \left\{ \frac{|x|^{3}}{(3)!}  ,  |x|^2 \right\}\le \min \left\{ |x|^{3}  ,  |x|^2 \right\}.
\]

This inequality will be used on each random variable in a row of the array. We will replace $x$ with particular $u U_{N,j}$.


\end{document}