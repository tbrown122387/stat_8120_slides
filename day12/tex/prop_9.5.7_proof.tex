\documentclass{article}

\usepackage{amsmath}
\usepackage{amsfonts, bbm}
\usepackage{url}
\usepackage{bbm} %for bold indicator func

\title{A More Detailed Proof of Proposition 9.5.7}

\begin{document}
\maketitle

The idea is we set $U_{N,i} = V_{N,i} - E[V_{N,i} \mid \mathcal{F}^N]$ and check conditons 1-3 of Proposition 9.5.5.
\newline

Condition 1: 
The conditional independence thing is obvious. Regarding the conditional expectations, first
\begin{align*}
E[|U_{N,i}| \mid \mathcal{F}^N] &= E[| V_{N,i} - E[V_{N,i} \mid \mathcal{F}^N] | \mid \mathcal{F}^N] \\
&\le E[| V_{N,i} | \mid \mathcal{F}^N] + E[ | E[V_{N,i} \mid \mathcal{F}^N]) | \mid \mathcal{F}^N] \tag{tri- ineq and linearity} \\
&= E[| V_{N,i} | \mid \mathcal{F}^N] + | E[V_{N,i} \mid \mathcal{F}^N]  | \\
&\le E[| V_{N,i} | \mid \mathcal{F}^N] +  E[ | V_{N,i}| \mid \mathcal{F}^N] \tag{Jensen's} \\
&< \infty \tag{assumption i}.
\end{align*}

Secondly,
\begin{eqnarray}
E[U_{N,i} \mid \mathcal{F}^N] = E[ V_{N,i} \mid \mathcal{F}^N] - E[V_{N,i} \mid \mathcal{F}^N] =0.
\end{eqnarray}

Condition 2: \\
We showed in conditon 1 that $E[|U_{N,i}| \mid \mathcal{F}^N] \le 2E[| V_{N,i} | \mid \mathcal{F}^N]$, so
\begin{eqnarray}
\sum_i E[|U_{N,i}| \mid \mathcal{F}^N] \le 2 \sum_i E[| V_{N,i} | \mid \mathcal{F}^N].
\end{eqnarray}
The right hand side is bounded in probability, so the left hand side is as well.
\newline


Condition 3: 

First, a little about the two tools used in this part, the verification of condition 3 in 9.5.5. The first trick is similar in spirit to that one used in 9.5.4. Recall that the triangle inequality gives us
\[
|U_{N,i}| \le |V_{N,i}| + E[|V_{N,i}| \mid \mathcal{F}^N]
\]
which implies (check contrapositive) that 
\[
\{ |U_{N,i}| \ge \epsilon\} \subset \{ |V_{N,i}| \ge \epsilon/2  \} \cup \{ E[|V_{N,i}| \mid \mathcal{F}^N] \ge \epsilon/2 \},
\]
which implies further that
\[
\mathbbm{1}_{\{ |U_{N,i}| \ge \epsilon\}} \le \mathbbm{1}_{\{ |V_{N,i}| \ge \epsilon/2  \} \cup \{ E[|V_{N,i}| \mid \mathcal{F}^N] \ge \epsilon/2 \}} \le \mathbbm{1}_{ \{ |V_{N,i}| \ge \epsilon/2  \}  } + \mathbbm{1}_{  \{ E[|V_{N,i}| \mid \mathcal{F}^N] \ge \epsilon/2 \}  }.
\]

The second trick is a way to bound the maximum. 
\begin{align*}
\max_{i} |V_{N,i}| &\le \max_{i} |V_{N,i}| \mathbbm{1}_{\{ |V_{N,i}| \ge \epsilon / 2 \} } + \max_{i} |V_{N,i}| \mathbbm{1}_{\{ |V_{N,i}| < \epsilon / 2 \} } \\
&\le \max_{i} |V_{N,i}| \mathbbm{1}_{\{ |V_{N,i}| \ge \epsilon / 2 \} } + \epsilon / 2 \\
&\le \sum_{i} |V_{N,i}| \mathbbm{1}_{\{ |V_{N,i}| \ge \epsilon / 2 \} } + \epsilon / 2.
\end{align*}

This will be used to say things like
$$
\{ \max_{i} |V_{N,i}|  \ge \epsilon \} \subseteq \left\{ \sum_{i} |V_{N,i}| \mathbbm{1}_{\{ |V_{N,i}| \ge \epsilon / 2 \} } \ge \epsilon/2 \right\}.
$$

Going back to the proof, we want to show that $\sum_i |U_{N,i}| \mathbbm{1}_{ \{|U_{N,i}| \ge \epsilon \} } $ converges to $0$ in probability:

\begin{align*}
&\sum_i E[ | U_{N,i} |  \mathbbm{1}(|U_{N,i}| \ge \epsilon) \mid \mathcal{F}^N] \\
&= \sum_i E[ | V_{N,i} - E[V_{N,i} \mid \mathcal{F}^N] |  \mathbbm{1}(|U_{N,i}| \ge \epsilon) \mid \mathcal{F}^N] \tag{defn.} \\
&\le \sum_i E[ | V_{N,i} - E[V_{N,i} \mid \mathcal{F}^N] |  \mathbbm{1}_{ \{ |V_{N,i}| \ge \epsilon/2  \}  } + \mathbbm{1}_{  \{ E[|V_{N,i}| \mid \mathcal{F}^N] \ge \epsilon/2 \}  } \mid \mathcal{F}^N] \tag{first trick} \\
&\le \sum_i E[ \left\{ | V_{N,i}| + |E[V_{N,i} \mid \mathcal{F}^N] | \right\} \left\{  \mathbbm{1}_{ \{ |V_{N,i}| \ge \epsilon/2  \}  } + \mathbbm{1}_{  \{ E[|V_{N,i}| \mid \mathcal{F}^N] \ge \epsilon/2 \}  } \right\} \mid \mathcal{F}^N]  \tag{tri-neq} \\
&\le \sum_i E[  | V_{N,i}|   \mathbbm{1}_{ \{ |V_{N,i}| \ge \epsilon/2  \}  } \mid \mathcal{F}^N] +
  2 \sum_i E[ | V_{N,i}| \mid \mathcal{F}^N] \mathbbm{1}_{  \{ E[|V_{N,i}| \mid \mathcal{F}^N] \ge \epsilon/2 \}  } \\
&\hspace{10mm} + \sum_i E[ | V_{N,i}|  \mid \mathcal{F}^N] P(|V_{N,i}| \ge \epsilon \mid \mathcal{F}^N) \\
&= \sum_i E[ | V_{N,i}|   \mathbbm{1}_{ \{ |V_{N,i}| \ge \epsilon/2  \}  } \mid \mathcal{F}^N] +  2B_N + A_N. \tag{defns in 9.69 and 9.70} \\
\end{align*}

The first term goes to $0$ by eqn. 9.6.8 in assumption (iii). This explains why we are to focus on proving 9.69 and 9.70. 
\newline

To proving $A_N \overset{p}{\to} 0$: 

\begin{align*}
A_n &= \sum_i E[ | V_{N,i}|  \mid \mathcal{F}^N] P(|V_{N,i}| \ge \epsilon/2 \mid \mathcal{F}^N ) \\
&\le P(\max_j |V_{N,j}| \ge \epsilon/2 \mid \mathcal{F}^N ) \sum_i E[ | V_{N,i}|  \mid \mathcal{F}^N] \tag{max} \\
&\le P\left( \left\{ \epsilon/4 + \sum_{j=1}^{M_N} |V_{N,j}| \mathbbm{1}(|V_{N,j}| \ge \epsilon/4) \right\}  \ge \epsilon/2 \bigg\rvert \mathcal{F}^N \right) \sum_i E[ | V_{N,i}|  \mid \mathcal{F}^N]  \tag{second trick} \\
&= P\left( \sum_{j=1}^{M_N} |V_{N,j}| \mathbbm{1}(|V_{N,j}| \ge \epsilon/4)   \ge \epsilon/4 \bigg\rvert \mathcal{F}^N \right) \sum_i E[ | V_{N,i}|  \mid \mathcal{F}^N]   \\
&\le (4/ \epsilon) \sum_j E\left[ |V_{N,j}| \mathbbm{1}(|V_{N,j}| \ge \epsilon/4) \mid \mathcal{F}^N \right]  \sum_i E[ | V_{N,i}|  \mid \mathcal{F}^N]  \tag{Markov's}.
\end{align*}
$\sum_i E\left[ |V_{N,i}| \mathbbm{1}(|V_{N,i}| \ge \epsilon/4) \mid \mathcal{F}^N \right]$ goes to $0$ by hypothesis. The other term is bounded in probability by hypothesis. Thus, the whole thing goes to $0$ by Lemma 9.5.3 (3).

Proving $B_N \overset{p}{\to} 0$: 
\newline

This uses the second inequality trick again:
\begin{align*}
B_N &=   \sum_i E[ | V_{N,i}|  \mid \mathcal{F}^N]  \mathbbm{1}( E[|V_{N,i}| \mid \mathcal{F}^N ] \ge \epsilon/2) \\
&\le  \mathbbm{1}( \max_i E[|V_{N,i}| \mid \mathcal{F}^N ] \ge \epsilon/2) \sum_i E[ | V_{N,i}|  \mid \mathcal{F}^N] \tag{max} \\
&\le  \mathbbm{1} \left\{ \sum_i E[|V_{N,i}| \mathbbm{1}(| V_{N,i} |\ge \epsilon/4) \mid \mathcal{F}^N ] \ge \epsilon/4 \right\}\sum_i E[ | V_{N,i}|  \mid \mathcal{F}^N]  \tag{second trick} \\
&\le (4/\epsilon) \sum_i E[|V_{N,i}| \mathbbm{1}(| V_{N,i} |\ge \epsilon/4) \mid \mathcal{F}^N ]  \sum_i E[ | V_{N,i}|  \mid \mathcal{F}^N] \tag{logic} \\
&\overset{p}{\to} 0
\end{align*}
where the last line follows because $\sum_i E[|V_{N,i}| \mathbbm{1}(| V_{N,i} |\ge \epsilon/4) \mid \mathcal{F}^N ]$ goes to $0$ by hypothesis, and the second is bounded in probability by hypothesis, which allows us to use Lemma 9.5.3(3) again. 

In the second to last line, I say ``logic" because this sum is nonnegative, and either it is bigger than $\epsilon/4$, or it isn't--verify the inequality holds in these two cases. 



\end{document}